\documentclass{article}

\usepackage{amsmath, amsthm, amssymb, amsfonts}
\usepackage{thmtools}
\usepackage{graphicx}
\usepackage{setspace}
\usepackage{geometry}
\usepackage{float}
\usepackage{hyperref}
\usepackage[utf8]{inputenc}
\usepackage[english]{babel}
\usepackage{framed}
\usepackage[dvipsnames]{xcolor}
\usepackage{tcolorbox}

%Define the listing package
\usepackage{listings} %code highlighter
\usepackage{color} %use color
\definecolor{mygreen}{rgb}{0,0.6,0}
\definecolor{mygray}{rgb}{0.5,0.5,0.5}
\definecolor{mymauve}{rgb}{0.58,0,0.82}
 
%Customize a bit the look
\lstset{ %
backgroundcolor=\color{white}, % choose the background color; you must add \usepackage{color} or \usepackage{xcolor}
basicstyle=\footnotesize, % the size of the fonts that are used for the code
breakatwhitespace=false, % sets if automatic breaks should only happen at whitespace
breaklines=true, % sets automatic line breaking
captionpos=b, % sets the caption-position to bottom
commentstyle=\color{mygreen}, % comment style
deletekeywords={...}, % if you want to delete keywords from the given language
escapeinside={\%*}{*)}, % if you want to add LaTeX within your code
extendedchars=true, % lets you use non-ASCII characters; for 8-bits encodings only, does not work with UTF-8
frame=single, % adds a frame around the code
keepspaces=true, % keeps spaces in text, useful for keeping indentation of code (possibly needs columns=flexible)
keywordstyle=\color{blue}, % keyword style
% language=Octave, % the language of the code
morekeywords={*,...}, % if you want to add more keywords to the set
numbers=left, % where to put the line-numbers; possible values are (none, left, right)
numbersep=5pt, % how far the line-numbers are from the code
numberstyle=\tiny\color{mygray}, % the style that is used for the line-numbers
rulecolor=\color{black}, % if not set, the frame-color may be changed on line-breaks within not-black text (e.g. comments (green here))
showspaces=false, % show spaces everywhere adding particular underscores; it overrides 'showstringspaces'
showstringspaces=false, % underline spaces within strings only
showtabs=false, % show tabs within strings adding particular underscores
stepnumber=1, % the step between two line-numbers. If it's 1, each line will be numbered
stringstyle=\color{mymauve}, % string literal style
tabsize=2, % sets default tabsize to 2 spaces
title=\lstname % show the filename of files included with \lstinputlisting; also try caption instead of title
}
%END of listing package%
 
\definecolor{darkgray}{rgb}{.4,.4,.4}
\definecolor{purple}{rgb}{0.65, 0.12, 0.82}
 
%define Javascript language
\lstdefinelanguage{JavaScript}{
keywords={typeof, new, true, false, catch, function, return, null, catch, switch, var, if, in, while, do, else, case, break},
keywordstyle=\color{blue}\bfseries,
ndkeywords={class, export, boolean, throw, implements, import, this},
ndkeywordstyle=\color{darkgray}\bfseries,
identifierstyle=\color{black},
sensitive=false,
comment=[l]{//},
morecomment=[s]{/*}{*/},
commentstyle=\color{purple}\ttfamily,
stringstyle=\color{red}\ttfamily,
morestring=[b]',
morestring=[b]"
}
 
\lstset{
language=JavaScript,
extendedchars=true,
basicstyle=\footnotesize\ttfamily,
showstringspaces=false,
showspaces=false,
numbers=left,
numberstyle=\footnotesize,
numbersep=9pt,
tabsize=2,
breaklines=true,
showtabs=false,
captionpos=b
}

\lstdefinelanguage{C}{
keywords={typeof, new, true, false, catch, function, return, null, catch, switch, var, if, in, while, do, else, case, break},
keywordstyle=\color{blue}\bfseries,
ndkeywords={class, export, boolean, throw, implements, import, this},
ndkeywordstyle=\color{darkgray}\bfseries,
identifierstyle=\color{black},
sensitive=false,
comment=[l]{//},
morecomment=[s]{/*}{*/},
commentstyle=\color{purple}\ttfamily,
stringstyle=\color{red}\ttfamily,
morestring=[b]',
morestring=[b]"
}

\colorlet{LightGray}{White!90!Periwinkle}
\colorlet{LightOrange}{Orange!15}
\colorlet{LightGreen}{Green!15}

\newcommand{\HRule}[1]{\rule{\linewidth}{#1}}

\NewEnviron{NORMAL}{% 
    \scalebox{2}{$\BODY$} 
} 

\declaretheoremstyle[name=Theorem,]{thmsty}
\declaretheorem[style=thmsty,numberwithin=section]{theorem}
\tcolorboxenvironment{theorem}{colback=LightGray}

\declaretheoremstyle[name=Proposition,]{prosty}
\declaretheorem[style=prosty,numberlike=theorem]{proposition}
\tcolorboxenvironment{proposition}{colback=LightOrange}

\declaretheoremstyle[name=Principle,]{prcpsty}
\declaretheorem[style=prcpsty,numberlike=theorem]{principle}
\tcolorboxenvironment{principle}{colback=LightGreen}

\setstretch{1.2}
\geometry{
    textheight=9in,
    textwidth=5.5in,
    top=1in,
    headheight=12pt,
    headsep=25pt,
    footskip=30pt
}

% ------------------------------------------------------------------------------

\begin{document}

% ------------------------------------------------------------------------------
% Cover Page and ToC
% ------------------------------------------------------------------------------

\title{ \normalsize \textsc{}
		\\ [2.0cm]
		\HRule{1.5pt} \\
		\LARGE \textbf{\uppercase{Reti di Calcolatori}
		\HRule{2.0pt} \\ [0.6cm] \LARGE{Corso A} \vspace*{10\baselineskip}}
		}

\date{\text{Ultima Compilazione - }\today}
\author{\textbf{Autore} \\ 
		Giuseppe Acocella \\
		2025/26\\
        \url{https://github.com/Peenguino}}

\maketitle
\newpage

\tableofcontents

\newpage

\section{Introduzione alle Reti}

Come facciamo a costruire una rete di computer che sia scalabile e che supporti
\textbf{svariati tipi di applicazioni} come streaming, messaggistica e videochiamate?
Internet ci permette di farlo, e tecnicamente è un insieme di miliardi di host (servers, laptops, smartphones)
connessi tra loro. 

\paragraph{Internet - Vista dei Componenti}

Questa connessione è permessa da specifici componenti:

\begin{enumerate}
    \item \textbf{Link di Comunicazione}: Fibra ottica, rame, onde radio, microonde.
    \item \textbf{Dispositivi di Interconnessione}: I più comuni sono:
    \begin{enumerate}
        \item \textbf{Switch}: Dispositivo che collega tra loro \textbf{più host}.
        \item \textbf{Router}: Dispositivo che collega tra loro \textbf{più reti}.
    \end{enumerate}
    \item \textbf{Reti}: Insieme di host, dispositivi di interconnessione e link gestiti da una stessa organizzazione.
\end{enumerate}

\paragraph{Internet - Vista dei Servizi} Da un punto di vista di servizi offerti invece
Internet:

\begin{enumerate}
    \item E' un \textbf{infrastruttura} che offre servizi alle \textbf{applicazioni distribuite}.
    \item Offre un \textbf{interfaccia di programmazione} (socket) che permette alle applicazioni, distribuite
    su \textbf{host diversi}, di scambiarsi informazioni.
\end{enumerate}

\paragraph{Internet - Vista delle Entità Software}

\begin{enumerate}
    \item \textbf{Applicazioni}: Elaborano e si scambiano informazioni.
    \item \textbf{Protocolli}: Regolano la trasmissione di queste informazioni, definendo
    quindi il formato e l'ordine dei messaggi scambiati tra due o più entità in comunicazione. Due esempi
    noti di protocolli internet sono:
    \begin{enumerate}
        \item \textbf{Trasmission Control Protocol (TCP)}
        \item \textbf{Internet Protocol (IP)}
    \end{enumerate}
\end{enumerate}

\paragraph{Standardizzazione di Protocolli} I protocolli di Internet vengono standardizzatida diversi enti, come:

\begin{enumerate}
    \item Internet Engineering Task Force (IETF). Si basano su una procedura detta RFC (Request for Comments)
    che seguendo passi rigorosi definisce gradualmente nuovi standard per Internet.
    \item IEEE 802 LAN Standards.
    \item World Wide Web Consortium (W3C): Comunità internazionale che sviluppa standard aperti per favorire
    lo sviluppo del Web.
\end{enumerate}

\newpage

\subsection{Tipi di Rete per Dimensione}

Descrizione dei tipi di rete per estensione fisica:

\begin{enumerate}
    \item \textbf{Personal Area Network} (PAN): Connessione personale, ad esempio una Bluetooth.
    \item \textbf{Local Area Network} (LAN): Reti circoscritte ad un area limitata, utilizzano cavi o mezzi wireless per connettersi e sono solitamente proprietarie.
    \item \textbf{Metropolitan Area Network} (MAN): Rete di calcolatori che non supera mai l'estensione fisica indicativa di una "città".
    \item \textbf{Wide Area Network} (WAN): Reti che cercano di interconnettere host separati da distanze geografiche tramite cavi in fibra ottica o ponti radio.
\end{enumerate}

\subsection{Gestione di Internet come Rete di Reti}

Ogni rete locale si espone alle altre reti grazie agli \textbf{Internet Service Provider} (ISP). Come possiamo immaginare però
non connettiamo ogni punto terminale ad ogni altro punto terminale perchè provocherebbe un alta complessità da un punto di vista di collegamenti.
Di conseguenza saranno presenti dei \textbf{Global ISP} che si occupano della gestione del route della comunicazione. Non esiste un solo \textbf{Global ISP} ma \textbf{molteplici}
e questi si connettono tra loro con dei \textbf{Peering Point} su punti d'incontro tra le reti detti \textbf{Internet Exchange Point} (IXP).

\paragraph{Content Provider Network} Idealmente si segue lo schema descritto sopra, ma a volte grandi aziende (Amazon, Google) si posizionano trasversalmente a questa organizzazione descritta, in modo tale da
gestire la propria rete per avvicinare i propri servizi all'utente finale.

\subsection{Reti e Mezzi d'Accesso}

Descriviamo questi elementi caratterizzanti dell'accesso ad una rete:

\begin{enumerate}
    \item \textbf{Rete d'Accesso}: Rete che connette fisicamente un sistema terminale al \textbf{edge router}, ossia il primo router da sorgente a destinazione.
    \item \textbf{Mezzi d'Accesso}: Come passiamo fisicamente dal calcolatore al edge router:
    \begin{enumerate}
        \item \textbf{Cablati (Vincolati)}: Doppino, coassiale, fibra.
        \item \textbf{Wireless (Non Vincolati)}: Onde Radio, microonde, infrarossi.
    \end{enumerate}
\end{enumerate}

\subsubsection{Schema dell'Accesso DSL}

Uno dei primi schemi d'accesso è stato quello Digital Subscriber Line (DSL), dove la compagnia telefonica rappresentava l'ISP.
Risultava quindi necessario splittare la linea telefonica già esistendo, suddividendo il traffico in bande di frequenze diverse:

\begin{enumerate}
    \item \textbf{Canale Telefonico}: $0-4 kHz$
    \item \textbf{Upstream}: $4-50 kHz$
    \item \textbf{Downstream}: $50 kHz \: - \: 1 MHz$
\end{enumerate}

\newpage

In questo modo sarà necessario un multiplexer di accesso alla linea digitale (DSLAM) che permetta di separare i due tipi di comunicazione. Esistono \textbf{versioni diverse}
di \textbf{questo schema}, riferiremo infatti allo schema \textbf{xDSL}, proprio perchè è possibile migliorare le caratteristiche della connessione ad esempio basandosi su una connessione
in \textbf{fibra} tra DSLAM ed ISP, ad esempio:

\begin{enumerate}
    \item \textbf{Fibre To The Cabinet (FTTC)}: Standard VDSL e VDSL2.
    \item \textbf{Fibre To The Derivation Point}.
    \item \textbf{Fibre To The House (FTTH)}: Caratterizzato dai seguenti elementi:
    \begin{enumerate}
        \item Fibra ottica fino all'interno delle abitazioni.
        \item Fibra uscente dalla centrale locale con terminatore ottico di linea \textbf{OLT} viene condivisa da diverse abitazioni.
        \item Il terminatore ottico di rete \textbf{ONT} invece è connesso ad uno splitter tramite fibra dedicata, questo permette la gestione fino al centinaio di abitazioni.
        \item L'\textbf{OLT} si connette al router dell'ISP e tramite questo ad Internet.
    \end{enumerate}
\end{enumerate}

\vspace*{10px}

\subsubsection{Tipi d'Accesso - Cavo, Aziendale, Domestico, Mobile e Satellitare}

Descriviamo queste tipologie di accesso alla rete:

\begin{enumerate}
    \item \textbf{Accesso Via Cavo}: Il segnale ottico viene convertito in segnale elettromagnetico e inviato sulle linee di cavi coassiali per la distribuzione del segnale
    alle varie abitazioni. La velocità potrebbe degradare a causa della distanza o della congestione del canale condiviso tra più abitazioni.
    \item \textbf{Accesso Aziendale}: Nelle aziende i sistemi sono collegati al router di frontiera con una LAN.
    \item \textbf{Accesso Domestico}: Utilizzate tecnologie Ethernet o WiFi per creare LAN domestiche.
    \item \textbf{Accesso Mobile}: Infrastruttura radio mobile definita dagli ISP, può subire degrado di prestazioni a causa di distanza oppure ostacoli.
    \item \textbf{Accesso Satellitare}: Accesso permesso da trasmettitori terrestri che si connettono a satelliti geostazionari (GEO) oppure satelliti a bassa quota (LEO).
    Questo sistema risulta essere il più duttile e non richiede complesse installazioni, ma potrebbe risultare inferiore da alcuni punti di vista come la latenza più alta.
\end{enumerate}

\newpage

\subsection{Nucleo della Rete - Commutazione Circuito vs Pacchetto}

Come determiniamo il percorso effettuato da un messaggio? Come trasferiamo dalla porta di uscita a quella di ingresso di due calcolatori? A queste domande
risponde il tipo di commutazione.

\paragraph{Commutazione di Circuito}

Si instaura un cammino tra gli host in comunicazione e tutte le risorse del canale sono dedicate a loro. Questo causa un utilizzo esclusivo del canale e conseguente spreco
di risorse. Per poter permettere molteplici comunicazioni utilizzando questo tipo di commutazione esistono due politiche:

\begin{enumerate}
    \item \textbf{Frequency Division Multiplexing (FDM)}: Divido il canale in bande di frequenze e le dedico alle diverse comunicazioni.
    \item \textbf{Time Division Multiplexing (TDM)}: Dedico tutto il canale alla coppia di host ma per una finestra di tempo limitata.
\end{enumerate}

\paragraph{Commutazione di Pacchetto}

Il flusso dati di una comunicazione viene suddiviso in \textbf{pacchetti}, e gli viene assegnato un \textbf{header} che mantenga dei metadati riguardanti il flusso dati originale.
Questo tipo di instradamento segue queste fasi:

\begin{enumerate}
    \item \textbf{Suddivisione} in \textbf{pacchetti} ed \textbf{assegnamento header}.
    \item \textbf{Instradamento} del singolo \textbf{pacchetto indipendentemente} dagli altri. Pacchetti di flussi dati diversi possono condividere i canali di comunicazione.
    \item Fase di \textbf{Store and Forward}, ossia prima che il commutatore (router) consegni i pacchetti dovrà prima aspettare che siano del tutto arrivati. Il tipo di Multiplexing
    quindi è statistico e non prefissato. Seguendo questa politica potremmo incorrere anche in problemi di \textbf{perdita di pacchetti} in base al tipo di gestione del buffer di ciascun commutatore.
    Non abbiamo quindi alcuna garanzia sulle prestazioni utilizzando questa tecnica. 
\end{enumerate}

\paragraph{Confronto Caratteristiche Commutazione Circuito/Pacchetto} Elenchiamo pro e contro di ciascuna tecnica:

\begin{enumerate}
    \item \textbf{Commutazione Circuito - Pro}:
    \begin{enumerate}
        \item Prestazioni garantite.
        \item Tecnologie di switching efficienti.
        \item Tariffazioni per ISP semplici da definire.
    \end{enumerate}
    \item \textbf{Commutazione Circuito - Contro}:
    \begin{enumerate}
        \item Segnalazione e configurazione delle tabelle di switching.
        \item Poca ottimizzazione del uso di risorse.
    \end{enumerate}
    \newpage
    \item \textbf{Commutazione Pacchetto - Pro}:
    \begin{enumerate}
        \item Utilizzo ottimale delle risorse.
        \item Non richiede fase di setup e tabelle di switching predefinite.
    \end{enumerate}
    \item \textbf{Commutazione Pacchetto - Contro}:
    \begin{enumerate}
        \item Tecnologie di inoltro non efficienti.
        \item Ritardi variabili.
        \item Rischio di congestione.
    \end{enumerate}

\end{enumerate}

\subsection{Metriche - Prestazioni della Rete (Latenza, Throughput)}

Descriviamo due metriche fondamentali delle prestazioni della rete:

\begin{enumerate}
    \item \textbf{Latenza}: Tempo richiesto dal primo bit partito dalla sorgente fino all'arrivo a destinazione. 
    Questa definizione è ideale, infatti dipende anche dal tipo di commutazione, quella di pacchetto con Store $\&$ Bound introduce altre difficoltà.
    Nello specifico le \textbf{cause specifiche} di \textbf{ritardo} nella \textbf{commutazione di pacchetto} sono:

    \begin{enumerate}
        \item \textbf{Ritardo di Elaborazione del Nodo}
        \begin{enumerate}
            \item Controllo errori sui bit.
            \item Determinazione porta di uscita.
        \end{enumerate}
        \item \textbf{Ritardo di Accodamento}
        
        Componente di ritardo più complessa da stabilire, si basa su una \textbf{condizione di stabilità} ossia un rapporto $\rho = \frac{L\:a}{R}$ dove $a$ è velocità
        media di arrivo, $R$ velocità di trasmissione ed $L$ lunghezza media dei pacchetti in bit.
        La condizione $\rho < 1$
        è detta \textbf{condizione di stabilità}.
        \begin{enumerate}
            \item Attesa di trasmissione.
            \item Dipende da intensità e tipo di traffico.
        \end{enumerate}
        \item \textbf{Ritardo di Trasmissione L/R}
        
        Tempo impiegato dal router per trasmettere un pacchetto sul link, con $R\:(bit/sec)$ indichiamo il rate, ossia la velocita di trasmissione del collegamento mentre con 
        $L \: (bit)$ la lunghezza del pacchetto. Il ritardo sarà quindi $d_{trasm} = \frac{L}{R}$. Quindi rappresenta il ritardo di passaggio nel commutatore.
        
        \item \textbf{Ritardo di Propagazione}
        
        Quanto un bit ci mette per spostarsi fisicamente, rappresenta il ritardo di passaggio nel link.
    \end{enumerate}

    Il ritardo complessivo è calcolato come somma di tutti i ritardi calcolati sopra.

\newpage

    \item \textbf{Throughput}
    Rappresenta la capacità effettiva di un link da host $A$ verso host $B$, difficile da definire formalmente, ma diamo questi due riferimenti:

    \begin{enumerate}
        \item \textbf{Throughput Istantaneo}: Velocità in $bit/sec$ a cui host $B$ riceve in ogni istante.
        \item \textbf{Throughput Medio}: Rapporto tra quantità $L$ di dati trasferiti e il tempo $T$ impiegato per il loro trasferimento, ovvero $\frac{L}{T}$.
    \end{enumerate}

    Un potenziale bottleneck influenzerebbe pesantemente questa metrica, in caso quindi di più di un collegamento $q_{i}$ il calcolo sarà $min(q_{1}, ...\:, q_{n})$.
    Il Throughput quindi è differente dalla velocità di trasmissione del collegamento che si pone come upper bound alla metrica del Throughput.

    \item \textbf{Prodotto Rate per Ritardo}
    Il prodotto rate per ritardo rappresenta il "volume" del collegamento, ossia il massimo numero di bit che possono riempire il collegamento ad
    un certo istante.

\end{enumerate}

\newpage

%\begin{figure}[htbp]
    %\center
    %\includegraphics[scale=0.4]{img/classiComplessita2.png}
%\end{figure}


\end{document}